\documentclass{sigchi}

% Arabic page numbers for submission. 
% Remove this line to eliminate page numbers for the camera ready copy
%\pagenumbering{arabic}

% Load basic packages
\usepackage{balance}  % to better equalize the last page
\usepackage{graphics} % for EPS, load graphicx instead
\usepackage{times}    % comment if you want LaTeX's default font
\usepackage{url}      % llt: nicely formatted URLs
\usepackage{todonotes}

% Load basic packages
%\usepackage[pdftex]{graphicx}
%\usepackage{amsmath, amssymb}
\usepackage{verbatim}
%\usepackage{tabularx}
\usepackage{mdwlist} %compact lists
%\usepackage{algorithm}
%\usepackage{algpseudocode} % program code

\usepackage{listings}

% llt: Define a global style for URLs, rather that the default one
\makeatletter
\def\url@leostyle{%
  \@ifundefined{selectfont}{\def\UrlFont{\sf}}{\def\UrlFont{\small\bf\ttfamily}}}
\makeatother
\urlstyle{leo}


% To make various LaTeX processors do the right thing with page size.
\def\pprw{8.5in}
\def\pprh{11in}
\special{papersize=\pprw,\pprh}
\setlength{\paperwidth}{\pprw}
\setlength{\paperheight}{\pprh}
\setlength{\pdfpagewidth}{\pprw}
\setlength{\pdfpageheight}{\pprh}

% create a shortcut to typeset table headings
\newcommand\tabhead[1]{\small\textbf{#1}}

% to give it a fighting chance of not being over-written, 
% since its job is to redefine many LaTeX commands.
\usepackage{ifpdf}
\ifpdf
\usepackage[pdftex]{hyperref}
\else
\usepackage{hyperref}
\fi
\hypersetup{
pdftitle={SIGCHI Conference Proceedings Format},
pdfauthor={LaTeX},
pdfkeywords={SIGCHI, proceedings, archival format},
bookmarksnumbered,
pdfstartview={FitH},
colorlinks,
citecolor=black,
filecolor=black,
linkcolor=black,
urlcolor=black,
breaklinks=true,
}

\graphicspath{{pictures/}}

% Use this command to override the default ACM copyright statement (e.g. for preprints). 
% Consult the conference website for the camera-ready copyright statement.
 \toappear{\scriptsize Permission to make digital or hard copies of all or part of this work for personal or classroom use is granted without fee provided that copies are not made or distributed for profit or commercial advantage and that copies bear this notice and the full citation on the first page. Copyrights for components of this work owned by others than the author(s) must be honored. Abstracting with credit is permitted. To copy otherwise, or republish, to post on servers or to redistribute to lists, requires prior specific permission and/or a fee. Request permissions from Permissions@acm.org. \\
 {\emph{UIST'13}}, October 8--11, 2013, St. Andrews, United Kingdom. \\
 Copyright \copyright~2013 ACM 978-1-4503-2268-3/13/10...\$15.00. \\
http://dx.doi.org/10.1145/2501988.2502007}
 
\clubpenalty=10000 
\widowpenalty = 10000 


\newcommand{\eg}{e.g.,\ }
\newcommand{\ie}{i.e.,\ }
\newcommand{\lablet}{Lablet\ }
\newcommand{\wysiwyg}{\mbox{WYSIWYG}\ }
\newcommand{\dragndrop}{drag and drop\ }


% End of preamble. Here it comes the document.
\begin{document}

\title{Lablet: A Mobile Learning Enviornment}

\numberofauthors{1}

\author{
	\alignauthor \mbox{Clemens Zeidler, Anna Yang, Kasper van Wijk}\\
		\affaddr{University of Auckland}\\
		\affaddr{38 Princes Street}\\
		\affaddr{Auckland 1010, New Zealand}\\
		\email{\{clemens.zeidler, a.yang, k.vanwijk\}@auckland.ac.nz}
}

\maketitle

%\tableofcontents

\begin{abstract}
%TODO Anna
\end{abstract}

\category{H.5.2}{User Interfaces}{Graphical user interfaces (GUI)}


\keywords{e-learning; mobile devices; physics experiments; electronic handouts}


\section{Introduction}
%Kasper: What do our (or any physics department's) labs aim to accomplish now (on paper), what does the lablet provide/add? Sensing, and analysis capabilities would be the two main things.
%A connection with "modern wired life" (there may be  a better term) is maybe another important consideration.

The increasing ownership and availability of mobile devices with
advanced sensors equip anyone interested in physics to get creative
with scientific inquiries.  While mobile technologies offer
opportunities to transform teaching and learning
practices~\cite{karnad2014trends, Kukulska2010}, relatively few mobile
projects have been designed that could facilitate social learning and
user-generated content~\cite{Frohberg2009}. 

Discuss \cite{Kearney2012,Etkina2006,Millar2002,Trumper2003}

The Auckland Lablet provides an active learning environment for
Physics experimentation on an Android tablet.  The intuitive interface
and the rich sensor suite of touch screen tablets create a powerful
learning tool built on commodity hardware.  Lablet users capture and
analyse data in and outside our teaching laboratories; Lablet modules
leverage a range of internal sensors in the tablet, and the fully
customisable Lablet environment can guide students through physics
experiments.

Lablet users optimise their time with learning hands-on Physics.  In
addition, Lablet creates a “paperless workflow” for grading and
student feedback, so that the Lablet maximises the time that lab
instructors spend interacting with students.

The goal of this project is to facilitate the growth of a community of
Lablet users in high schools, studio-based learning
environments, and flipped classrooms.  Lablet’s low cost and
versatility create a wealth of opportunities for playful independent
exploration.  
%Imagine students combining Doppler analysis and video
%motion tracking data of cars on a busy road, using an accelerometer in
%an elevator to measure the height of a building, or analysing the
%motion of a ball on a sport field.

The Auckland Lablet is open source; our applications and source code
are freely available for use and modification.  The current release
includes video, audio and accelerometer capturing and analysis
modules.  Future releases will exploit the ambient light sensors,
compasses, gyroscopes and GPS receivers found in tablets, as well as
purpose-built Arduino-based Bluetooth connected sensors.

Next we introduce the internal tablet sensors supported in Lablet as a
new Mobile learning environment, and the analysis capabilities that
accompany the sensing. We then discuss possible experiments, with data
acquired by supported sensors and analysed on the tablet. Finally, we
present the scripting language of Lablet to link experiments with
individual sensors or embed one or more experiments in what we call
``activities'' in Lablet.


\subsection{Contributions}
Our contributions can be summarised as follows:

\begin{enumerate}
\item \lablet, a novel physics learning environment for mobile devices.
\item Lab Activities.
\end{enumerate}

\section{Related Work}
% Web camera as a measuring tool in the undergraduate physics laboratory and Investigating viscous damping using a webcam 
For example, a web camera has been used to analyze diffusion of ink drop in water, damped oscillations of a pendulum, diffraction patterns~\cite{Nedev2006} and mass oscillating in a viscous fluid~\cite{Shamim2010}.
% Cellular Phones Helping To Get a Clearer Picture of Kinematics 
The kinematic of a water jet from a hose can be studied by taking a photo of the scene using a mobile device~\cite{Falcao2009}. \lablet users have the option to record video at different resolutions and
frame rates.

Lablet's motion analysis module is inspired by tracker\todo[inline]{cite from the webpage Conference Presentations...}\footnote{\url{http://physlets.org/tracker/}}.
However, tracker does only supports one sensor and an external camera is needed to record a video.
% sound:

% acoustics
Using a fast Fourier transformation the different types of sounds such as a tone, a periodic and none-sinusoidal vibration pattern, noise and impulse can be analyzed using the microphone of a mobile device~\cite{KuhnAcousticPhenomena2013}.

% accustic measurement of a bouncing ball
Mobile devices can also be used to record the sound of an bouncing ball~\cite{Schwarz2013Acoustic}.
From the time difference of the bounces and the initial height the gravitation can be determined.

% bell-jar
The sound propagation in vacum can be demonstrated by hanging a ringing mobile phone into bell-jar~\cite{CaleonBellJar2013}.


% accelerometer:

% Analyzing free fall with a smartphone acceleration sensor
Another experiment proposes to facilitate the accelerometer of a mobile device to analyze a free fall by dropping the mobile device on a cushion~\cite{VogtFreeFall2012}.
% Analyzing spring pendulum phenomena and radial acceleration with a smartphone acceleration sensor
Similar, a mobile device can be used to record acceleration of a spring pendulum, a coupled spring pendulum~\cite{KuhnPendulum2012}, a free and a damped harmonic oscillation~\cite{Castro2013}.
By rotating the mobile device the radial acceleration can be measured~\cite{VogtRadialAcc2013}.


% other sensors:

% The optical mouse for harmonic oscillator experimentation
An optical mouse can be used as an extremely cost-effective displacement sensor, for example, to measure the damped harmonic oscillations~\cite{Ng2005}.

% oscilloscope
Mobile devices can be act a portable oscilloscopes~\cite{Forinash2012}.
The microphone jack can be used as an input.
The recorded data can be Fourier analyzed.

% Smartphones—Experiments with an External Thermistor Circuit
The microphone jack can also be used as an input for external sensors such as a thermistor~\cite{Forinash2012}.

% Wiimote
The Wiimote sensor can been used to track multiple objects simultaneously, \eg to observe a spring system with multiple coupled masses~\cite{Skeffington2012}.

% A simple demonstration for exploring the radio waves generated by a mobile phone
A mobile phone can also be used as a radio wave source that can be measured using a simple wave detector~\cite{Hare2010}.


\section{Sensors}
\lablet uses the build in sensors of mobile devices to record data.
Data from multiple sensors can be recorded in parallel, for example, to record the an visual and acoustic data of an event.
Currently, these sensors are the camera, the microphone and the accelerometer.
The quality and abilities of a sensor depends on the manufacture and the used model.
For this reason we only discuss the general properties of these sensors and discuss possible limitations based on our experiences using \lablet in the lab classes.

\paragraph{Camera}
Modern mobile devices are usually equipped with at least one camera that can record videos at a reasonable high resolution.
In general cameras in mobile devices do not record at a fix recording frame rate but only try to reach a target frame rat.
From our experiences many cameras are able to record with a reasonable stable frame rate between 25 and 30 fps.
However, a recording frame rate of 30fps is not uncommon.
In other words experiments can be recorded with a time resolution of 34-40ms.
\todo[inline]{is that called time resolution?}
% I don't know what exactly going on so I'm a bit vage here and do not name a certain component for the problem:
Failing to record a video frame usually results in duplicated frames in the video file, \ie the missing frame is compensated by another frame.

% blur of fast moving objects
A common problem with cameras in mobile devices is that fast moving object become blurry in the video.
This can effect how precise the position of a moving object in a video can be detected.
Usually a good illumination of the recorded scene can help to reduce the blur effect.

% long videos
Recording long videos can result in very large video files that might exceed the storage of the mobile device.
For example, recording a video at 1280x800@30fps results in a mp4 file roughly of the size of 4.2 Gb per hour.
In order to reduce the disk requirements \lablet allows to lower the recording frame rate or video resolution.
\lablet allows to record video with a video frame rate down to 0.1 fps but theoretically there is now lower limit for the recording frame rate.
This allows to record long video with either a high video- and a low time-resolution or with a low video- and a high time-resolution.

\paragraph{Microphone}
Common microphones in mobile devices are able to record audio data with a sampling rate of 44100Hz.
This allows the un-aliased recording of sound from roughly 20Hz to 22000Hz which is corresponds to the hearing range of humans.
Note, since microphones in mobile devices are designed to record the human audible spectrum a microphone might be less sensitive for certain frequency ranges.
For example, high frequencies are less important for hearing than low frequencies.

\paragraph{Accelerometer}
Accelerometer in mobile devices usually detect accelerations in x, y and z direction.
Since there are many different accelerometers used in mobile devices we just give some example specifications to illustrate the capabilities of such accelerometers.
With the Galaxy Tab4 we were able to record $\sim100$ accelerometer data points per second.
The BMA180\footnote{\url{http://www.sensorsmag.com/product/mems-accelerometer-bosch-sensortec}}, a microelectromechanical system (MEMS) accelerometer can detect accelerations of $0.00025g$ and a tilt change in the field of gravity of $0.25^{\circ}$

%Lablet calculate a resulting total acceleration value from these three
%\[
%a_{total} = \sqrt{a_x^2 + a_y^2 + a_z^2}
%\]

\section{Analysis Modules}
Recorded sensor data can directly be analysed using Lablet's analysis modules.
Moreover, external video or audio data can be imported to \lablet and then be analysed within \lablet.
Currently, there are three analysis modules available that are presented in the following.

\subsection{Motion Analysis Module}
The motion analysis module can be used to analyse a moving object in a recorded video.
The main idea here is to step throw the video frame by frame and mark the position of the moving object.
Using the so gathered object position data in combination with the time data from the video frames the the velocity and acceleration of the moving object can be derived.

%\subsubsection{Motion Tracking}
\lablet supports to automatically track an object in a recorded video.
After marking an object in a video frame \lablet is able to mark this object automatically in the subsequent frames.
Object tracking works best for object with high contrast to their background.

% video settings
\lablet allows to configure what part of the video should be analysed, \ie the video can be cropped to an interesting region.
Furthermore, the user can choose at what frame rate the video should be analysed.
For example, if only every second frame needs to be analysed the analysis frame rate can be halved (see Figure~\ref{fig:MotionAnalysisSettingsScreen}).

\begin{figure}
  \centering
  \includegraphics[width=.99\columnwidth]{MotionAnalysisSettings}
  \caption{Motion analysis settings screen.
  On the left bottom the part of the video that contains the experiment can be selected.
On the right side the analysis frame rate can be adjusted.\label{fig:MotionAnalysisSettingsScreen}}
\end{figure}

% length scale
One problem that arise when analysing a moving object in a video is that in general the distance of the object to the camera is unknown and thus the travelled distance of the object can not be determined.
To solve this problem, \lablet assumes a reference object with known size that is placed in the recorded scene.
By marking the reference object using a linear scale and by specifying the length of the scale \lablet can derive the spacial extend of the recorded scene (see Figure~\ref{fig:MotionAnalysis}).
% how to get optimal results
Note, that for optimal results the video should be recorded parallel to the plane of the moving object and the reference object should be located in that plane.

% origin
Optionally, the coordinate system of the recorded scene can be defined, \ie the position and the orientation of the coordinate system can be specified (Figure~\ref{fig:MotionAnalysis}).
Furthermore, the $x$ and $y$ axes can be swapped.
This can, for example, be helpful to analyse object that are moving from right to left.

\begin{figure}
  \centering
  \includegraphics[width=.99\columnwidth]{MotionAnalysis}
  \caption{Motion analysis screen.
  The green length scale can be used to calibrate the size of the video scene.
  The coordinate system can be used to set origin and orientation of the video scene.\label{fig:MotionAnalysis}}
\end{figure}

\subsection{Frequency Analysis Module}
The frequency analysis module performs a discrete Fourier analysis on recorded audio data.
This can be used to discuss the frequency spectrum of a sound source.
For example, to analyse the Doppler effect of a moving sound source.
The data from the Fourier analysis is visualized in a frequency map that displays the frequency data versus the recording time (see Figure~\ref{fig:FrequencyAnalysis})
The frequency analysis module allows to change various parameter of the discrete Fourier analysis which makes it a good tool to demonstrate a discrete Fourier analysis to students.

% cursors
In order to highlight interesting points in the frequency map the frequency analysis module allows to position an arbitrary number horizontal and vertical cursors.
This can, for example, be used to measure the period of an audio signal or the frequency difference of
a Doppler shift experiment.

% visualization
To visualize very small frequency changes the user can changing the colour contrast and brightness.
Furthermore, the frequency map is zoomable and makes it possible to magnify interesting data points.

% details
In the following details about the performed discrete Fourier analysis are presented.
To perform a discrete Fourier analysis on the recorded audio data the audio data is first partitioned into fix sized data windows.
For each window first a Hamming window function and then a discrete Fourier analysis is performed.
The discrete Fourier analysis is then performed on each of these windows.
The window size as well as the overlap of two consecutive windows can be configure.
For example, a window overlap of 0\% means that two consecutive windows are directly adjacent while an overlap of 50\% means that the second half of the previous window is the first half of the following window.


\begin{figure}
  \centering
  \includegraphics[width=.99\columnwidth]{FrequencyAnalysis}
  \caption{Frequency analysis screen.
  The results of a discrete Fourier analysis is displayed versus the recording time.\label{fig:FrequencyAnalysis}}
\end{figure}

% what window size and window overlap do
At a sampling rate $f_s$ and a window size $n$ samples time between two windows is
\[
\Delta t_w = n / f_s
\]
This means the time resolution of the time frequency plot is high if
the window size is small.  When using a percental window overlap $p$
this time deceases further to
\[
\Delta t_w = n / f_s * (1 - p / 100)
\]
Note, that when using a window overlap only existing data is reused
thus no new information is used when increasing the window overlap.
However, since a hamming window function is indeed sense to use a
window overlap and in practice some frequency changes become more
visible at certain window sizes.

% frequency resolution
% TODO: is frequency resolution the right term?
The frequency resolution is given by
\[
\Delta f = f_s / (n - 1)
\]
Thus, the trade-off of increasing the time resolution by decreasing
the window size is that the frequency resolution degrades.  In
practice a good compromise has to be found that gives good results for
a certain problem.

\subsection{Accelerometer Analysis Module}
Lablet automatically integrates the measured accelerometer data to get
velocity and displacement of the device.  To do so we assume the
mobile device is initially at rest and only gravity is affecting it.
Usually the measured total acceleration at rest is not equal to the
earth gravity $g$ and the value has to be calibrated.  The
experimenter can do this by placing a calibration line in the time
acceleration graph (Figure~\ref{fig:AccelerometerAnalysis}).

% problem
Integrating the acceleration is a very difficult task and small errors
in the initial acceleration results in a drift of the integrated data.
This problem is exasperated in the second integral.  Because of that
one can only expect qualitative results for velocity and distance.

% elevator example and discussion
The accelerometer data in Figure~\ref{fig:AccelerometerAnalysis} shows
a journey in an elevator from the top of the building to the ground
floor.  The first peek shows how the elevator accelerates till it
reach a certain speed then the second peek indicates how the elevator
decelerates to come to a stop at the bottom of the building.  If one
would place the calibration line in the time acceleration graph
without looking at the integrals it is almost certain that velocity
and distance graphs have a large drift.  To avoid this problem one can
use the knowledge that the lift was at rest at the top and at the
bottom of the journey.  Fiddling with the calibration line till the
velocity graph shows a velocity of zero at the beginning and the end
results in a travelled distance that is actually compatible with the
height of the building.

\begin{figure}
  \centering
  \includegraphics[width=.99\columnwidth]{AccelerometerAnalysis}
  \caption{Accelerometer analysis screen.  In the top time
    acceleration the experimenter can calibrate the initial
    acceleration at rest.  From this the velocity and distance graphs
    are calculate (bottom left and bottom
    right).\label{fig:AccelerometerAnalysis} }
\end{figure}

\subsection{Data Export}


\section{Lab Activities}
In a classical laboratory class students usually get a paper-based
handout that guides them through the class.  The handout can include
instructions, questions and other tasks.  For example, the handout can
instruct students to manually calculate the velocity of a moving
object using Lablet's motion analysis module.  The desired learning
outcome for this experiment could be to understand how to calculate
the velocity from consecutive position measurements and discuss the
velocity time graph.  To do so data can be exported from the mobile
device and analyzed on a computer.  However, this complicates the
laboratory class since more equipment is required and students needs
to learn to use the analysis software.  Another way to analysis the
data is to calculate the velocity manually using pen, paper and a
calculator.  The drawback of this method is that students have to
repeat a relative simple task multiple time and the results and
because teachers have to recalculate the results they are difficult to
verify.  Either way, students have to spend time with tasks that are
not part of the desired learning outcome.

% Lablet's solution
Lablet solves this shortcomings described above by providing a way to
run {\em Lab Activities} tailored for individual laboratory classes.
A Lab Activity can include a multitude of elements such as
instructions, questions, experiments, visualization of experiment
data or advanced analysis task.

\begin{figure}
  \centering
  \includegraphics[width=.99\columnwidth]{LabActivitySheet}
  \caption{Sheet from a Lab Activity containing various different
    elements.}
  \label{fig:LabActivitySheet} 
\end{figure}

% desciption of the Lab Activities UI
Each Lab Activity can have multiple sheets.  The sheets are in a
horizontal layout and students can go to next or previous sheet using
a swipe gesture.  Each sheet can have an unlimited number of elements.
Some elements requires to be completed before students can proceed to
the next sheet.  For example, check boxes needed to be ticked before
the next sheet becomes available.  Figure~\ref{fig:LabActivitySheet}
shows an example of a single sheet containing various different
elements.

% layout
Lablet comes with a powerful but simple layout model to position
elements on a sheet.  Elements can be placed in horizontal or vertical
boxes.  Vertical and horizontal layouts can be nested to create more
complex layouts.  For example, the main layout in
Figure~\ref{fig:LabActivitySheet} is a vertical layout while the two
graphs are placed in a horizontal layout.

\subsection{Default Elements}
In the following a brief description of the available elements that
can be placed on a sheet is given.

\paragraph{Header}
The header element can be used to mark a new section on the sheet.

\paragraph{Text}
The text element displays some text and can be used for instructions
or other informations.

\paragraph{Check Box}
The check box element has some desciptive text and a check box.  All
check boxes on a sheets need to be checked before the next sheet is
activated.

\paragraph{Question}
The question element is a text only question.

\paragraph{Question with Answer Text}
This element is the same as the basic question element but it has a
field for text input.

\paragraph{Sensor Experiment}
The experiment element let students start an experiments.  There are
currently three versions of this element, \ie on version each for the
camera, microphone and accelerometer sensor.

\paragraph{Data Analysis}
The experiment analysis element let students start an analysis of data
recorded with the experiment element.  As for the experiment there are
currently three different types of experient analyses, \ie motion
analysis, frequency analysis and accelerometer analysis.

\subsection{Motion Analysis Elements}
So far we used Lablet mainly for motion analysis and thus there are
some special elements that are described in the following.

\paragraph{Graph}
The graph element uses data from the motion analysis element to
display a various interessting plots.  A axes of a graph element can
be configured to show differen data.  A axis can show time, position,
velocity and acceleration data.

\paragraph{Derivation}
The Derivation element is an excercise for students to calculate
velocity and acceleration from taken data.  Two velocity points have to
be calculated from three successive data points to calculate
the acceleration value from the two velocity points.  Moreover,
students have to select the correct unit for velocity and
acceleration.  There is an automatic check that varifies each step of
the calculation.  This makes it easy for demonstrators to spot
problems students have with the calculation.  Once the calculation for
the sample points has been done correctly by the students the rest of
the values are calculated and displayed to the students.

\paragraph{Potential Energy}
This element let the students estimate the potential energy in Joule
at a certain point in there experiment.  This students then have
calculate what the equivalent in calories is and compare this value to
the calories of an example piece of food.  Again, if the correctness
of the calculation is verified by the system.

\subsection{Editing Lab Activities}
From the techniqual point of view Lab Activities are simple text files
that describe the sheets and the contents.  The text file can easily
imported and managed on the mobile device.  The following listing
shows a small example containing a single sheet and some basic
elements.

\begin{lstlisting}
Lablet = {
    interface = 1.0,
    title = "Lab Activity Demo Sheet"
}
 
function Lablet.buildActivity(builder) {
    -- comment: add a single sheet
    sheet = builder:create("Sheet")
    builder:add(sheet)
    sheet:setTitle("Physics Laboratory")
    sheet:addHeader("Lab equipment:")
    sheet:addText("Do you have:")
    sheet:addCheckQuestion("metre rule")
    sheet:addCheckQuestion("ball")
}
\end{lstlisting}


\begin{comment}
\section{Experiments}
In this section we summarize a possible set of experiments that can be
conducted using Lablet.

\section{Improved Learning Environment}
% TODO Anna:
- face to face 
- avoiding repeatative tasks

\section{Evaluation}

\subsection{Observations}
Students use calculators and pen and paper to fill speed and
accelerator values.  That is good because they real have to get into
the problem; Lablet does not do the core part for you it is just a
tool that helps learning.
\end{comment}


\section{Conclusion}

\paragraph{Future Work}
- more sensors/ external sensors
- Combined analysis, \eg camera and mic
- Make it easier for teachers to create lab activies


\bibliographystyle{acm-sigchi}
\bibliography{BibEntries}

\end{document}




